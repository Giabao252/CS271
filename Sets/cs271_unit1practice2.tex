\documentclass[11pt, letter]{article}
\usepackage{geometry}
\usepackage{graphicx}
\usepackage{amsmath}
\usepackage{fancyhdr, lastpage}
\pagestyle{fancyplain}
\fancyhf{}
\lhead{ \fancyplain{}{CS271: Data Structures} }
\rhead{ \fancyplain{}{Fall 2023} }
\fancyfoot[RO, LE] {page \thepage\ of \pageref{LastPage} }
\thispagestyle{plain}

\begin{document}
\begin{center}
{\Large \textsc{CS271: Data Structures}}\vspace{\baselineskip}\\

Name: YOUR NAMES HERE\vspace{\baselineskip}\\
Instructor: Dr. Stacey Truex
\vspace{\baselineskip}\\
\Large{Unit 1: Practice 2}
\end{center}

\section*{Asymptotic Proofs}

Prove the following using the definitions of $O$, $\Omega$, and $\Theta$.
    
    \begin{enumerate}
    
    \item $100n + 10000 = O(n^2)$\\

    Want to show that $\exists$ positive constants $c, n_0$ so that:
    
    \begin{center}
    $0 \leq 100n + 10000 \leq cn^2$ $\forall n \geq n_0$
    \end{center}
    
    \begin{equation*}
        \begin{split}
            100n + 10000 &> 0 \forall n \geq 0\\
            \\
            100n + 10000 &\leq 2n^2\\
            \Leftrightarrow 2n^2 - 100n - 10000 &\geq 0\\
            \Leftrightarrow n &\geq 100
        \end{split}
    \end{equation*}
    
    For constants $c = 2, n_0 = 100$, $0 \leq 100n + 10000 \leq cn^2$ $\forall n \geq n_0$. Therefore, $100n + 10000 = O(n^2)$.

    \item $n^3 + 22n + 6 = O(n^3)$\\

    Want to show that $\exists$ positive constants $c, n_0$ so that:

    \begin{center}
    $0 \leq n^3 + 22n + 6 \leq cn^3$ $\forall n \geq n_0$
    \end{center}
    
    \begin{equation*}
        \begin{split}
            n^3 + 22n + 6 &> 0 \forall n \geq 0\\
            \\
            n^3 + 22n + 6 &\leq 2n^3\\
            \Leftrightarrow n^3 - 22n - 6 &\geq 0\\
            \Leftrightarrow n &\geq 5
        \end{split}
    \end{equation*}
    
    For constants $c=2$, $n_0 = 5$, $0 \leq n^3 + 22n + 6 \leq cn^3$ $\forall n \geq n_0$. Therefore, $n^3 + 22n + 6 = O(n^3)$.

    \item $3n^2 - 5n + 12 = \Omega(n^2)$\\

    Want to show ...
    
    \begin{equation*}
        \begin{split}
            ? &= ?\\
            &> ?
        \end{split}
    \end{equation*}
    
    For constants $\epsilon \geq 1$, $?$, $...$. Therefore, $3n^2 - 5n + 12 = \Omega(n^2)$.

    \item $n^\epsilon = \Omega(\log_{10} n)$ for any constant $\epsilon \geq 1 $\\

    Want to show ...
    
    \begin{equation*}
        \begin{split}
            ? &= ?\\
            &> ?
        \end{split}
    \end{equation*}
    
    For constants $\epsilon \geq 1$, $?$, $...$. Therefore, $n^\epsilon = \Omega(\log_{10} n)$.

    \item $3n^3 + 6n - 3 = \Theta(n^3)$\\

    Want to show ...
    
    \begin{equation*}
        \begin{split}
            ? &= ?\\
            &< ?
        \end{split}
    \end{equation*}
    
    Therefore for constants $?$, $...$. We also show:
    
    \begin{equation*}
        \begin{split}
            ? &= ?\\
            &> ?
        \end{split}
    \end{equation*}
    
    For constants $?$, $...$. Therefore, $3n^3 + 6n - 3 = \Theta(n^3)$.

    \item $5\log_2 n + 8\log_{16} n = \Theta(\log_2 n)$\\

    Want to show ...
    
    \begin{equation*}
        \begin{split}
            ? &= ?\\
            &< ?
        \end{split}
    \end{equation*}
    
    Therefore for constants $?$, $...$. We also show:
    
    \begin{equation*}
        \begin{split}
            ? &= ?\\
            &> ?
        \end{split}
    \end{equation*}
    
    For constants $?$, $...$. Therefore, $5\log_2 n + 8\log_{16} n = \Theta(\log_2 n)$.

\end{enumerate}

\section*{Recurrence Proofs}

For each of the following recurrences, find a tight upper bound for $T(n)$. You may find the formulas sheet available on the Canvas homepage useful. Then prove that your bound is correct using induction.  In each case, assume that $T(n)$ is constant for $n \leq 2$ and that floor division applies to all recurrences.
\vspace{\baselineskip}
\begin{enumerate}
\item $T(n) = T(n-1) + bn$ for some positive constant $b$\\

    \begin{center}
        Guess: $T(n) = $ ??\\
    \end{center}
    
    Inductive Hypothesis: \\
    Inductive Step:\\
    \begin{equation*}
        \begin{split}
            ? &= ?\\
            &< ?
        \end{split}
    \end{equation*}
    
    Let $a > 0$ be the constant runtime when $n\leq 2$. Base Case(s):
    \begin{itemize}
        \item ?
    \end{itemize}  
    
    For constants $?$ $...$. Therefore, $...$.\\
    
\item $T(n) = T(n/2) + b$ for some positive constant $b$\\

    \begin{center}
        Guess: $T(n) = $ ??\\
    \end{center}
    
    Inductive Hypothesis: \\
    Inductive Step:\\
    \begin{equation*}
        \begin{split}
            ? &= ?\\
            &< ?
        \end{split}
    \end{equation*}
    
    Let $a > 0$ be the constant runtime when $n\leq 2$. Base Case(s):
    \begin{itemize}
        \item ?
    \end{itemize}  
    
    For constants $?$ $...$. Therefore, $...$.\\
    
\item $T(n) = 8T(n/8) + bn^2$\\

    \begin{center}
        Guess: $T(n) = $ ??\\
    \end{center}
    
    Inductive Hypothesis: \\
    Inductive Step:\\
    \begin{equation*}
        \begin{split}
            ? &= ?\\
            &< ?
        \end{split}
    \end{equation*}
    
    Let $a > 0$ be the constant runtime when $n\leq 2$. Base Case(s):
    \begin{itemize}
        \item ?
    \end{itemize}  
    
    For constants $?$ $...$. Therefore, $...$.\\
    
\item $T(n) = T(n/8) + bn^2$\\

    \begin{center}
        Guess: $T(n) = $ ??\\
    \end{center}
    
    Inductive Hypothesis: \\
    Inductive Step:\\
    \begin{equation*}
        \begin{split}
            ? &= ?\\
            &< ?
        \end{split}
    \end{equation*}
    
    Let $a > 0$ be the constant runtime when $n\leq 2$. Base Case(s):
    \begin{itemize}
        \item ?
    \end{itemize}  
    
    For constants $?$ $...$. Therefore, $...$.\\
    
\item $T(n) = 5T(n/8) + bn$\\

    \begin{center}
        Guess: $T(n) = $ ??\\
    \end{center}
    
    Inductive Hypothesis: \\
    Inductive Step:\\
    \begin{equation*}
        \begin{split}
            ? &= ?\\
            &< ?
        \end{split}
    \end{equation*}
    
    Let $a > 0$ be the constant runtime when $n\leq 2$. Base Case(s):
    \begin{itemize}
        \item ?
    \end{itemize}  
    
    For constants $?$ $...$. Therefore, $...$.\\
    
\end{enumerate}

\end{document}
